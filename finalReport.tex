\documentclass[12pt,letterpaper]{article}
\usepackage{anysize}
\usepackage{indentfirst}
\usepackage{sectsty}
\usepackage{amsmath}
\usepackage{hyperref}
\usepackage{graphicx}
\usepackage{chngpage}
\usepackage{enumerate}
\hypersetup{
	colorlinks=true, 
	linkcolor=blue, 
	urlcolor=blue, 
	pdfnewwindow=true, 
	citecolor=black
}
\urlstyle{same}
\linespread{1.2}


\makeatletter
\renewcommand\@biblabel[1]{\textbf{#1.}} % Change the square brackets for each bibliography item from '[1]' to '1.'
\renewcommand{\@listI}{\itemsep=0pt} % Reduce the space between items in the itemize and enumerate environments and the bibliography


\begin{document}

\begin{titlepage}
    \vspace*{4cm}
    \begin{flushright}
    {\huge
        Quad Control\\[5mm]
    }
    {\large
        CS 472 | Spring 2015
     }
    \end{flushright}
\hrule
    \begin{flushright}
	Audrey Sullivan\\
	Vedanth Narayanan\\
    \vfill
	\today\\
    \end{flushright}
\end{titlepage}

\raggedright

\section*{Introduction}

\section*{Concept Design}
We propose an original wearable device that is capable of controlling a quadcopter. It would be very convenient to be able to direct a droid with a wristband-like device, with vertical and horizontal directing controls. This device would simplify functionality and be a great alternative to large handheld controllers.
Our design utilizes an obtuse isosceles triangular shape that allows you to control with your thumb and index fingers. You can essentially use two wide sides of a triangle with the obtuse angle on top to control different directions at the same time while the third side is attached to a wristband. \\
Our idea involves sending the copter commands to the app, and have the app take care of being able to send the commands to the quad copter. Some of the apps that exist today include Quadroid for Android and Astrodrone for Android/iOS. \\
MIT has been created a wearable device called NailO, which allows you to control other devices from your nail bed. The device works by sticking it to a nail (usually thumb) like you would a nail sticker and using another finger on that hand to drag on that surface. We are using similar principles in our design to obtain two NailO like pads on the two faces of the triangular design. \\

\subsection*{Assumptions}
\begin{itemize}
\item Assume the app communicating with our device has secure features such as a secure bluetooth connection that stops other apps from controlling the same quadcopter. 
\item The device, unlike conventional wearables, will not always be on. It’s meant to only be turned on when it needs to be used. 
\item To be able to properly piggyback on an existing  mobile application, there are definitely changes that need to be made, including being able to connect to the device through Bluetooth, and properly parsing the commands from the device, and then passing them off to the quad copter.
\end{itemize}

\subsection*{Use Cases}
\begin{itemize}
\item I want to be able to integrate Quad Control with my Quadroid App.
\begin{itemize}
\item Our assumptions are that you can use Quad Control as an assumption of an existing smart phone app.
\item We will have room for extra capabilities in case more wants to be added to the Quad Control.
\end{itemize}
\item I want to be able to click to change settings from Quad Control.
\begin{itemize}
\item You can use the touch pads as buttons with a tap.
\end{itemize}
\item I want to...
\end{itemize}



\section*{Specifications}
\subsection*{Pros and Cons}
Battery consumption?
Wearable has a small battery, so charge it more often.


\section*{Conclusion}

\newpage
\nocite{*}
\bibliographystyle{annotate}
\bibliography{bib}

\end{document}
